%!TEX root = ../RelazioneStrutturaleMeoliNicola.tex
\chapter{Trave P13-P18-Ascensore -- Piano primo}
%!TEX root = ../RelazioneStrutturaleMeoliNicola.tex
\begin{figure}[htb]
\centering
\begin{tikzpicture}
\scaling{0.55};
	\point{n1}{00.00}{0};
	\point{n2}{03.00}{0};
	\point{n3}{07.50}{0};
	\point{n4}{11.50}{0};
	\point{n5}{16.50}{0};
	\point{n6}{22.65}{0};
	\point{n7}{26.65}{0};
	\beam{2}{n1}{n2}[0][1];
	\beam{2}{n2}{n3}[1][1];
	\beam{2}{n3}{n4}[1][1];
	\beam{2}{n4}{n5}[1][1];
	\beam{2}{n5}{n6}[1][1];
	\beam{2}{n6}{n7}[1][1];
	\begin{scope}[scale=0.6]
		\support{1}{n1};
		\support{1}{n2};
		\support{1}{n3};
		\support{1}{n4};
		\support{1}{n5};
		\support{1}{n6};
		\support{3}{n7}[90];
	\end{scope};	
	\dimensioning{1}{n1}{n2}{-1.5}[$\SI{3.00}{\meter}$];
	\dimensioning{1}{n2}{n3}{-1.5}[$\SI{4.50}{\meter}$];
	\dimensioning{1}{n3}{n4}{-1.5}[$\SI{4.00}{\meter}$];
	\dimensioning{1}{n4}{n5}{-1.5}[$\SI{5.00}{\meter}$];
	\dimensioning{1}{n5}{n6}{-1.5}[$\SI{6.15}{\meter}$];
	\dimensioning{1}{n6}{n7}{-1.5}[$\SI{4.00}{\meter}$];
	%Nomi dei punti
	\notation{4}{n1}{n2}[$1$];
	\notation{4}{n2}{n3}[$2$];
	\notation{4}{n3}{n4}[$3$];
	\notation{4}{n4}{n5}[$4$];
	\notation{4}{n5}{n6}[$5$];
	\notation{4}{n6}{n7}[$6$];
	%
	\point{m1}{00.00}{-1.5};
	\point{m2}{03.00}{-1.5};
	\point{m3}{07.50}{-1.5};
	\point{m4}{11.50}{-1.5};
	\point{m5}{16.50}{-1.5};
	\point{m6}{22.65}{-1.5};
	\point{m7}{26.65}{-1.5};
	\notation{6}{m1}{$1$};
	\notation{6}{m2}{$2$};
	\notation{6}{m3}{$3$};
	\notation{6}{m4}{$4$};
	\notation{6}{m5}{$5$};
	\notation{6}{m6}{$6$};
	\notation{6}{m7}{$7$};
\end{tikzpicture}
\caption{Schema strutturale adottato e relativa nomenclatura utilizzata}
\label{fig:Struttura0}
\end{figure}
\section{Analisi dei carichi trave}
\subsection{Peso proprio trave}
La trave è di sezione rettangolare $30 \times \SI{50}{\centi\metre}$ in calcestruzzo armato. 
La normativa in \normaref{Tab. 3.1.II} suggerisce di utilizzare un peso specifico $\gamma_{CLS}$ pari a \SI{25.0}{\kilo\newton\per\meter\cubed} per il calcestruzzo armato. 
Pertanto il carico lineare risulta 
\[
	G_1^{trave} = 0.3 \cdot 0.5 \cdot 25 = \SI{3.75}{\kilo\newton\per\meter}
\]

\subsection{Terrazzo}
\paragraph*{Carichi permanenti G1}
Come da progetto il peso del solaio ultimato in travetti tralicciati in latero cemento è pari a $g_1^{ter.}=\SI{3.20}{\kilo\newton\per\square\meter}$.
\paragraph*{Carichi permanenti non strutturali G2}
Il carico distribuito su superficie permanente non strutturale agente sul terrazzo è la somma dei singoli carichi degli strati che compongono la stratigrafia presente.
\begin{center}
\begin{tabular}{lS[table-format=2.1]S[table-format=1.2]S[table-format=1.3]}
	\toprule
	\multirow{2}{*}{Strato} & \multicolumn{1}{c}{Peso specifico} & \multicolumn{1}{c}{Spessore}& \multicolumn{1}{c}{$g_{2,k}$}\\
    	   & \multicolumn{1}{c}{$\left[\SI{}{\kilo\newton\per\meter\cubed}\right]$} & \multicolumn{1}{c}{$\left[\SI{}{\meter}\right]$}& \multicolumn{1}{c}{$\left[\SI{}{\kilo\newton\per\square\meter}\right]$}\\
	\midrule
	Isolante 	             & 0.5  & 0.15 & 0.075 \\
	Massetto calcestruzzo 	 & 24.0 & 0.06 & 1.44  \\
	Pavimento 	             &      &      & 0.50  \\
	Intonaco intradosso 	 & 20.0 & 0.01 & 0.20 \\
	\midrule
	Totale $g_2^{ter.} =$ &&&2.215\\
	\bottomrule
\end{tabular}
\end{center}
\paragraph*{Categoria B - balconi} Si è supposto che la funzione del terrazzo sia equiparabile strutturalmente a quella di un balcone. 
La \normaref{Tab. 3.1.II} delle \norma{NTC2018} prevede per i balconi un carico distribuito pari a \SI{4.00}{\kilo\newton\per\square\meter}
\paragraph*{Neve}
L'edificio è ubicato in zona 1 ad una quota superiore a \SI{200}{\meter} pertanto il valore di riferimento del carico della neve al suolo risulta pari a
\[
	q_{sk}=1.39 \, [1+(a_s/728)^2] = \SI{1.626}{\kilo\newton\per\square\meter}
\]

Si assume che l'edificio sia in zona normale di vento. Pertanto $C_E$ risulta pari a $1$.
Si assume un coefficiente termico $C_t = 1$ in quanto è assente uno specifico studio riguardo la perdita di calore della costruzione. 

\begin{figure*}[htb]
\centering
\includegraphics{IMG/figC3-4-5.pdf}
\label{fig:C345}
\end{figure*}
Per il calcolo del coefficiente di forma $\mu_i$ la \norma{circolare} in \normaref{Fig. C.3.4.5} (riportata a pagina \pageref{fig:C345}) prevede due possibili casi dovuti alla vicinanza della copertura a costruzioni più alte in quanto si genera un accumolo di neve.
Il primo caso prevede $\mu_1=0.8$ ed è costante data la copertura piana. Nel secondo caso $\mu_2$ è la somma tra il contributo $\mu_s$ dello scivolamento della neve dalla copertura al piano superiore e pertanto è nullo essendo piana anch'essa. 
E il contributo $\mu_w$ dovuto al vento che redistribuisce la neve. 
Questo vale $\mu_w=\frac{b_1 + b_2}{2\,h}$: si hanno quindi due casi dovuti alla diversa dimensione di $b_2$ che vale \SI{6.00}{} e \SI{3.50}{\meter} rispettivamente tra le zone indicate in FIGURA DA METTERE.
Gli altri termini invece valgono $b_1=\SI{18.00}{\meter}$ e $h=\SI{6.20}{\meter}$. 
Si ottiene $\mu_w^1=1.935$ e $\mu_w^2=1.734$.

Essendo $l_s=2\,h>b_2$ in entrambi i casi, il coefficiente $\mu$ deve essere interpolato in base alla lunghezza $b_2$. 
Usando la similitudine dei triangoli come mostrato in FIGURA DA METTERE si ottiene l'altezza del triangolo a distanza $b_2$ dall'edificio e che sommato al valore dell'altezza $\mu_1$ del rettangolo porta al valore del \normaref{caso 2} cercato 
\begin{align*}
	\mu_2^{1}=&\mu_1 + \frac{(l_s - b_2^1)\cdot (\mu_w^1-\mu_1)}{l_s} = 1.386\\
	\mu_2^{2}=&\mu_1 + \frac{(l_s - b_2^2)\cdot (\mu_w^2-\mu_1)}{l_s} =	1.470
\end{align*}
Per semplicità si assume ora un valore unico del coefficiente tra i valori delle altezze del trapezio del caso 2 e pari alla media tra $\mu_2$ e $\mu_w$ ottenendo $\mu^1= 1.661$ e $\mu^2=1.602$ e costante.

Il carico dovuto alla neve sul terrazzo risulta infine pari a 
\begin{align}
q_s^1 &= q_{sk} \cdot C_E \cdot C_t \cdot \mu^1 = \SI{1.626}{} \cdot 1 \cdot 1 \cdot 1.661 = \SI{2.700}{\kilo\newton\per\square\meter}\\
q_s^2 &= q_{sk} \cdot C_E \cdot C_t \cdot \mu^2 = \SI{1.626}{} \cdot 1 \cdot 1 \cdot 1.602 = \SI{2.605}{\kilo\newton\per\square\meter}\label{eq:qneve}
\end{align}
\paragraph*{Vento} \label{cap:ventoTerrazzo} 
La velocità base di riferimento $V_b$ è data dalla \normaref{3.3.1} delle \norma{NTC2018} e dalla relativa \normaref{Tab. 3.3.1} nel quale il coefficiente di altitudine $c_a$ vale $1$ perché la quota è inferiorie ad $a_0$ e $V_{b,o}=\SI{25}{\meter\per\second}$. 
Il coefficiente di ritorno $c_r$ della \normaref{formula 3.3.2} è pari a $1$ nel caso di un tempo pari a 50 anni. 
Pertanto la velocità di riferimento $V_r=\SI{25}{\meter\per\second}$.
Assumendo una densità dell'aria $\rho$ come consigliato nel \normaref{\S 3.3.6}, la pressione cinetica di riferimento vale $q_r =1/2\, \rho\, V_r^2 = \SI{0.39}{\kilo\newton\per\square\meter}$. 

L'edificio è ubicato in provincia di Trento pertanto è in zona urbana ad una quota inferiore a \SI{500}{\meter} ed ad una distanza maggiore di \SI{30}{\kilo\meter} dal mare. 
Risulta quindi dalla \normaref{Tab. 3.3.III} una classe di rugosità del terreno A e dalla \normaref{Fig. 3.3.2} una classe di esposizione V del sito.
Pertanto dalla \normaref{Tab. 3.3.II} si ha
\[
	k_r=0.23 \qquad z_0=\SI{0.70}{\meter} \quad z_{min}=\SI{12}{\meter}
\]
Il coefficiente di topografia $c_t$ è assunto pari a $1$.
La quota del terrazzo in cui si sta calcolando l'azione del vento è pari a \SI{3.50}{\meter}.
Il coefficiente di esposizione risulta pari alla \normaref{formula 3.3.7} 
\[
	c_e(z)=c_e(z_{min})=k_t^2\cdot c_t \cdot \ln(z/z_0)\cdot[7+c_t\cdot\ln (z/z_0)] = 1.48
\]

Per il calcolo del coefficiente di pressione $c_p$ si è preso come riferimento il \normaref{\S C3.3.8.1.2} riguardante le coperture piane non essendo elencati nelle normative casi specifici per i terrazzi.

Si sono utilizzati i coefficienti globali in quanto si vuole calcolare la pressione o depressione complessiva esercitata dalla forza del vento. 
Si avranno due valori di pressione positivi e negativi e che verranno usati per ottenere poi (nel paragrafo \ref{cap:combinazioniCarico}) i valori sfavorevoli e favorevoli.

Per il calcolo del coefficiente $C_{pe}$ la \norma{circolare} propone la distinzione di due zone A e B, la prima avente dimensione $\min\{ b/2\,;\,h\}$ che nel caso in esame è pari a $\min\{\frac{\SI{33.65}{\meter}}{2};\SI{3.50}{\meter}\}=\SI{3.50}{\meter}$.
Essendo quindi il terrazzo delimitato da entrambe le zone e non volendo calcolare la forza totale del vento ma il carico su superficie, si è considerato nel caso di pressione il coefficiente positivo $C_{pe,B}=+0.2$ da usare nel caso sfavorevole e il coefficiente negativo $C_{pe,B}=-0.20$ da usare nel caso favorevole di depressione.
Quest'ultimo in realtà a favore di sicurezza in quanto sarebbe da considerare anche il coefficiente $C_{pe,A}=-0.80$.

Il coefficiente dinamico $c_d$ è preso pari ad 1 come suggerito nel capitolo \normaref{\S 3.3.9}

Pertanto il carico dovuto al vento è pari a 
\[
	q_w = q_r \cdot c_e \cdot c_p \cdot c_d = \SI{0.39}{\kilo\newton\per\square\meter}\cdot 1.48 \cdot \pm 0.20 \cdot 1= \pm \SI{0.1154}{\kilo\newton\per\square\meter}
\]
\subsection{Interno}
\paragraph*{Carichi permanenti strutturali G1}
\e presente il medesimo solaio strutturale del terrazzo, pertanto $g_1^{sol.}=\SI{3.20}{\kilo\newton\per\square\meter}$.
\paragraph*{Carichi permanenti non strutturali G2}\label{cap:g2Trave} Sono costituiti dal pacchetto non strutturale della stratigrafia del solaio e dalle pareti divisorie interne. 
Per quanto riguarda le pareti divisorie interne il \normaref{\S 3.1.3} delle \norma{NTC2018} permette di spalmare il peso delle pareti interne in un carico distribuito su tutta la superficie.
\begin{center}
\begin{tabular}{lS[table-format=2.1]S[table-format=1.2]S[table-format=1.3]}
	\toprule
	\multirow{2}{*}{Strato} & \multicolumn{1}{c}{Peso specifico} & \multicolumn{1}{c}{Spessore}& \multicolumn{1}{c}{$g_{2,k}$}\\
    	   & \multicolumn{1}{c}{$\left[\SI{}{\kilo\newton\per\meter\cubed}\right]$} & \multicolumn{1}{c}{$\left[\SI{}{\meter}\right]$}& \multicolumn{1}{c}{$\left[\SI{}{\kilo\newton\per\square\meter}\right]$}\\
	\midrule
	Tramezze in laterizio 	 	 & 8.00 & 0.08 & 0.64 \\
	Intonaco interno 	     	 & 20.0 & 0.01 & 0.2 \\
	Intonaco esterno	         & 20.0 & 0.01 & 0.2 \\
	\midrule
	Totale $=$   				 &      &      & 1.04 \\
	\bottomrule
\end{tabular}
\end{center}
L'altezza delle pareti corrisponde all'altezza di interpiano meno lo spessore del solaio, il che risulta $\SI{3.10}{} - \SI{0.25}{} = \SI{2.85}{\meter}$.
Il carico lineare delle pareti interne diviene quindi \SI{2.964}{\kilo\newton\per\meter}.
Utilizzando la normativa si ottiene così un carico di \SI{1.20}{\kilo\newton\per\square\meter}.

Unendo tutti i contributi si ha 
\begin{center}
\begin{tabular}{lS[table-format=2.1]S[table-format=1.2]S[table-format=1.3]}
	\toprule
	\multirow{2}{*}{Strato} & \multicolumn{1}{c}{Peso specifico} & \multicolumn{1}{c}{Spessore}& \multicolumn{1}{c}{$g_{2,k}$}\\
    	   & \multicolumn{1}{c}{$\left[\SI{}{\kilo\newton\per\meter\cubed}\right]$} & \multicolumn{1}{c}{$\left[\SI{}{\meter}\right]$}& \multicolumn{1}{c}{$\left[\SI{}{\kilo\newton\per\square\meter}\right]$}\\
	\midrule
	Sottofondo CLS alleggerito 	 & 16.0 & 0.08 & 1.28 \\
	Massetto allettamento 	     & 24.0 & 0.06 & 1.44 \\
	Pavimento ceramica 	         &      &      & 0.50 \\
	Intonaco intradosso 	     & 20.0 & 0.01 & 0.20 \\
	Pareti interne distribuite   &      &      & 1.20 \\
	\midrule
	Totale $g_2^{sol.} =$        &      &      & 4.62 \\
	\bottomrule
\end{tabular}
\end{center}
\paragraph*{Categoria B - Uffici} La \normaref{Tab. 3.1.II} prevede un carico variabile di \SI{3.00}{\kilo\newton\per\square\meter} per la categoria uffici.
\subsection{Pareti perimetrali}
Sono agenti direttamente con un carico lineare al di sopra della trave e si estendono per una altezza pari a quella di interpiano meno la trave.
Ovvero $\SI{3.10}{} - \SI{0.5}{} = \SI{2.60}{\meter}$.
\begin{center}
\begin{tabular}{lS[table-format=2.1]S[table-format=1.2]S[table-format=1.3]}
	\toprule
	\multirow{2}{*}{Strato} & \multicolumn{1}{c}{Peso specifico} & \multicolumn{1}{c}{Spessore}& \multicolumn{1}{c}{$g_{2,k}$}\\
    	   & \multicolumn{1}{c}{$\left[\SI{}{\kilo\newton\per\meter\cubed}\right]$} & \multicolumn{1}{c}{$\left[\SI{}{\meter}\right]$}& \multicolumn{1}{c}{$\left[\SI{}{\kilo\newton\per\square\meter}\right]$}\\
	\midrule
	Muratura in laterizio 	 	 & 10   & 0.30 & 3 \\
	Intonaco interno 	     	 & 20.0 & 0.01 & 0.2 \\
	Cappotto esterno	         & 0.20 & 0.12 & 0.024 \\
	\midrule
	Totale $=$   				 &      &      & 3.224 \\
	\bottomrule
\end{tabular}
\end{center}
\[
G_2^{pareti}=\SI{3.224}{\kilo\newton\per\square\meter} \cdot \SI{2.60}{\meter} = \SI{8.382}{\kilo\newton\per\meter}
\]
\section{Totale carichi agenti sulla trave}
Vengono ora moltiplicati i risultati appena trovati per le relative lunghezze di influenza. 
Si è diviso il problema in tre zone: A, B e C come mostrato in FIGURA DA METTERE. 
Nella zona A e B si ha una zona esterna dell'edificio e una zona interna.
Pertanto si considera il carico gravante sulla trave in oggetto di calcolo spalmato sui $5/8$ della luce nella zona verso l'esterno e alla metà della luce nella zona sottostante.  
Diviene rispettivamente pari a 
\begin{align*}
	\text{A :}&\qquad L^{ter.}=\frac{5}{8}\cdot\SI{6.00}{\meter}=\SI{3.75}{\meter} \qquad 
				L^{sol.} \frac{1}{2}\cdot\SI{5.00}{\meter} = \SI{2.50}{\meter}\\
	\text{B :}&\qquad L^{ter.}=\frac{5}{8}\cdot\SI{3.50}{\meter}=\SI{2.19}{\meter} \qquad 
				L^{sol.} \frac{1}{2}\cdot\SI{5.00}{\meter} = \SI{2.50}{\meter}\\
\end{align*}
Nella zona C invece l'orditura del solaio del terrazzo al primo piano è nell'altra direzioni. 
A tal proposito si considera una lunghezza di influenza del terrazzo simbolica di \SI{1.00}{\meter}.
Nella parte sottostante è uguale a quella delle altre zone.
\[
	\text{C :}\qquad L^{ter.}=\SI{1.00}{\meter} \qquad 
				L^{sol.} \frac{1}{2}\cdot\SI{5.00}{\meter} = \SI{2.50}{\meter}
\]

I carichi a metro lineare sotto riportati tenendo conto di tali lunghezze e sono stati combinati con le relativi azioni agenti sulla trave.

Nel caso dei sovraccarichi variabili si è assunta l'ipotesi che essi agiscano sempre insieme. 
Ovvero quando è possibile che avvenga il valore caratteristico nel terrazzo, questo avverrà anche nel solaio interno.
\paragraph*{Zona A} 
\[
\begin{split}
G_1^A &=  g_1^{ter.}\cdot L^{ter.} + g_1^{sol.}\cdot L^{sol.} + G_1^{trave} \\
&= \SI{3.20}{\kilo\newton\per\square\meter}\cdot\SI{3.75}{\meter} + \SI{3.20}{\kilo\newton\per\square\meter}\cdot\SI{2.50}{\meter} + \SI{3.75}{\kilo\newton\per\meter} \\
&= \SI{23.75}{\kilo\newton\per\meter} \\
G_2^A &= g_2^{ter.}\cdot L^{ter.} + g_2^{sol.}\cdot L^{sol.} + G_2^{pareti} \\
&= \SI{2.215}{\kilo\newton\per\square\meter}\cdot\SI{3.75}{\meter} + \SI{4.62}{\kilo\newton\per\square\meter}\cdot\SI{2.50}{\meter} + \SI{8.382}{\kilo\newton\per\meter} \\
&= \SI{28.24}{\kilo\newton\per\meter}\\
Q_{cat. B}^A &= q_{cat. B}^{ter.}\cdot L^{ter.} + q_{cat. B}^{sol.}\cdot L^{sol.} \\
&= \SI{4.00}{\kilo\newton\per\square\meter}\cdot\SI{3.75}{\meter} + \SI{3.00}{\kilo\newton\per\square\meter}\cdot\SI{2.50}{\meter} \\
&= \SI{22.50}{\kilo\newton\per\meter}\\
Q_{neve}^A &= q_s \cdot L^{ter.} = \SI{2.700}{\kilo\newton\per\square\meter}\cdot\SI{3.75}{\meter} = \SI{10.13}{\kilo\newton\per\meter}\\
Q_{vento}^A &= q_w \cdot L^{ter.} = \SI{\pm 0.1154}{\kilo\newton\per\square\meter}\cdot\SI{3.75}{\meter} = \SI{\pm 0.4328}{\kilo\newton\per\meter}
\end{split}
\]
\paragraph*{Zona B} I carichi su superficie sono gli stessi della zona A ma cambiano le lunghezze di riferimento e il carico della neve $q_s$ come visto nella \eqref{eq:qneve}. 
Si ha perciò
\begin{align*}
G_1^B &= \SI{18.76}{\kilo\newton\per\meter}\\
G_2^B &= \SI{24.78}{\kilo\newton\per\meter}\\
Q_{cat. B}^B &=  \SI{16.26}{\kilo\newton\per\meter}\\
Q_{neve}^B &= \SI{5.705}{\kilo\newton\per\meter}\\
Q_{vento}^B &= \SI{\pm 0.2527}{\kilo\newton\per\meter}
\end{align*}
\paragraph*{Zona C} Si hanno gli stessi carichi della zona A con la relativa lunghezza di riferimento
\begin{align*}
G_1^C &= \SI{14.95}{\kilo\newton\per\meter}\\
G_2^C &= \SI{22.15}{\kilo\newton\per\meter}\\
Q_{cat. B}^C &=  \SI{11.50}{\kilo\newton\per\meter}\\
Q_{neve}^C &= \SI{2.700}{\kilo\newton\per\meter}\\
Q_{vento}^C &= \SI{\pm 0.1554}{\kilo\newton\per\meter}
\end{align*}
\section{Combinazioni di carico}\label{cap:combinazioniCarico}
Al fine di trovare le azioni più incisive nel caso di carico massimo e di carico minimo, si sono valutate le azioni sfavorevoli e favorevoli con diverse disposizione nelle campate. 
Si elencheranno qui le diverse possibili combinazioni di carico agli stati limite ultimi e di esercizio.
\paragraph*{Zona A}
\allowdisplaybreaks %Serve per fare andare le equazioni su più pagine. Non spezza gli split perché non hanno questa possibilità (meglio). Per spezzare in un punto specifico \displaybreak
\begin{align} 
	\begin{split}
	SLU^{\text{sfav}}_{\text{cat. B}} &= \gamma_{G1}\cdot G_1 + \gamma_{G2} \cdot G_2 + \gamma_{cat. B} \cdot Q_{cat. B} + \gamma_{neve}\cdot Q_{neve}\cdot\psi_{02} + \gamma_{vento}\cdot Q_{vento} \cdot \psi_{03}  \\
	&= 1.3\cdot\SI{23.75}{} + 1.5\cdot\SI{28.24}{} + 1.5\cdot\SI{22.50}{} + 1.5\cdot\SI{10.13}{}\cdot0.5 + 1.5\cdot\SI{0.4328}{}\cdot0.6\\
	&= \SI{115.0}{\kilo\newton\per\meter}
	\end{split} \\ 
	\begin{split}
	SLU^{\text{sfav}}_{\text{neve}} &= \gamma_{G1}\cdot G_1 + \gamma_{G2} \cdot G_2 + \gamma_{neve}\cdot Q_{neve} + \gamma_{cat. B} \cdot Q_{cat. B}\cdot\psi_{02} + \gamma_{vento}\cdot Q_{vento} \cdot \psi_{03}  \\
	&= 1.3\cdot\SI{23.75}{} + 1.5\cdot\SI{28.24}{} + 1.5\cdot\SI{10.13}{} + 1.5\cdot\SI{22.50}{}\cdot0.7 + 1.5\cdot\SI{0.4328}{}\cdot0.6\\
	&= \SI{112.4}{\kilo\newton\per\meter}
	\end{split} \\ 
	\begin{split}
	SLU^{\text{sfav}}_{\text{vento}} &= \gamma_{G1}\cdot G_1 + \gamma_{G2} \cdot G_2 + \gamma_{vento}\cdot Q_{vento} + \gamma_{cat. B} \cdot Q_{cat. B}\cdot\psi_{02} + \gamma_{neve}\cdot Q_{neve} \cdot \psi_{03}  \\
	&= 1.3\cdot\SI{23.75}{} + 1.5\cdot\SI{28.24}{} + 1.5\cdot\SI{0.4328}{} + 1.5\cdot\SI{22.50}{}\cdot0.7 + 1.5\cdot\SI{10.13}{}\cdot0.5\\
	&= \SI{105.1}{\kilo\newton\per\meter}
	\end{split} \\ 
	\begin{split}
	SLU^{\text{fav}} &= \gamma_{G1}\cdot G_1 + \gamma_{G2} \cdot G_2 + \varnothing\\
	&= 1.0\cdot\SI{23.75}{} + 0.8\cdot\SI{28.24}{}\\
	&= \SI{46.34}{\kilo\newton\per\meter}	
	\end{split} \\ 
	\begin{split}
	SLE^{\text{rara}}_{\text{cat. B}} &= G_1 + G_2 + Q_{cat. B} + \psi_{02}\cdot Q_{neve} + \psi_{03}\cdot Q_{vento}  \\
	&= \SI{23.75}{} + \SI{28.24}{} + \SI{22.50}{} + 0.5\cdot\SI{10.13}{} + 0.6\cdot\SI{0.4328}{}\\
	&= \SI{79.81}{\kilo\newton\per\meter}
	\end{split} \\ 
	\begin{split}
	SLE^{\text{rara}}_{\text{neve}} &= G_1 + G_2 + Q_{neve} + \psi_{02}\cdot Q_{cat. B} + \psi_{03}\cdot Q_{vento}  \\
	&= \SI{23.75}{} + \SI{28.24}{} + \SI{10.13}{} + 0.7\cdot\SI{22.50}{} + 0.6\cdot\SI{0.4328}{}\\
	&= \SI{78.13}{\kilo\newton\per\meter}
	\end{split} \\ 
	\begin{split}
	SLE^{\text{rara}}_{\text{vento}} &= G_1 + G_2 + Q_{vento} + \psi_{02}\cdot Q_{cat. B} + \psi_{03}\cdot Q_{neve}  \\
	&= \SI{23.75}{} + \SI{28.24}{} + \SI{0.4328}{} + 0.7\cdot\SI{22.50}{} + 0.6\cdot\SI{10.13}{}\\
	&= \SI{74.25}{\kilo\newton\per\meter}
	\end{split} \\ 
	\begin{split}
	SLE^{\text{frequente}}_{\text{cat. B}} &= G_1 + G_2 + \psi_{11}\cdot Q_{cat. B} + \psi_{22}\cdot Q_{neve} + \psi_{23}\cdot Q_{vento}  \\
	&= \SI{23.75}{} + \SI{28.24}{} + 0.5\cdot\SI{22.50}{} + \varnothing +\varnothing\\
	&= \SI{63.24}{\kilo\newton\per\meter}
	\end{split} \\ 
	\begin{split}
	SLE^{\text{frequente}}_{\text{neve}} &= G_1 + G_2 + \psi_{11}\cdot Q_{neve} + \psi_{22}\cdot Q_{cat. B} + \psi_{23}\cdot Q_{vento}  \\
	&= \SI{23.75}{} + \SI{28.24}{} + 0.2\cdot\SI{10.13}{} + 0.3\cdot\SI{22.50}{} + \varnothing \\
	&= \SI{60.77}{\kilo\newton\per\meter}
	\end{split} \\ 
	\begin{split}
	SLE^{\text{frequente}}_{\text{vento}} &= G_1 + G_2 + \psi_{11}\cdot Q_{vento} + \psi_{22}\cdot Q_{cat. B} + \psi_{23}\cdot Q_{neve}  \\
	&= \SI{23.75}{} + \SI{28.24}{} + 0.2\cdot\SI{0.4328}{} + 0.3\cdot\SI{22.50}{} + \varnothing\\
	&= \SI{58.83}{\kilo\newton\per\meter}
	\end{split} \\ 
	\begin{split}
	SLE^{\text{quasi perm.}}_{\text{cat. B}} &= G_1 + G_2 + \psi_{21}\cdot Q_{cat. B} + \psi_{22}\cdot Q_{neve} + \psi_{23}\cdot Q_{vento} \\
	&= \SI{23.75}{} + \SI{28.24}{} + 0.3\cdot\SI{22.50}{} + \varnothing + \varnothing \\
	&= \SI{58.74}{\kilo\newton\per\meter}
	\end{split} 
\end{align}
\paragraph*{Zona B} Analogamente
\begin{align*} 
	SLU^{\text{sfav}}_{\text{cat. B}}		&= \SI{90.23}{\kilo\newton\per\meter} \\
	SLU^{\text{sfav}}_{\text{neve}} 		&= \SI{87.42}{\kilo\newton\per\meter} \\
	SLU^{\text{sfav}}_{\text{vento}} 		&= \SI{83.29}{\kilo\newton\per\meter} \\
	SLU^{\text{fav}} 						&= \SI{38.58}{\kilo\newton\per\meter} \\	
	SLE^{\text{rara}}_{\text{cat. B}} 		&= \SI{62.80}{\kilo\newton\per\meter} \\
	SLE^{\text{rara}}_{\text{neve}}			&= \SI{60.78}{\kilo\newton\per\meter} \\
	SLE^{\text{rara}}_{\text{vento}} 		&= \SI{58.60}{\kilo\newton\per\meter} \\
	SLE^{\text{frequente}}_{\text{cat. B}} 	&= \SI{51.67}{\kilo\newton\per\meter} \\
	SLE^{\text{frequente}}_{\text{neve}} 	&= \SI{49.56}{\kilo\newton\per\meter} \\
	SLE^{\text{frequente}}_{\text{vento}} 	&= \SI{48.47}{\kilo\newton\per\meter} \\
	SLE^{\text{quasi perm.}}_{\text{cat. B}}&= \SI{48.42}{\kilo\newton\per\meter}
\end{align*}
%
\paragraph*{Zona C} Analogamente
\begin{align*} 
	SLU^{\text{sfav}}_{\text{cat. B}}		&= \SI{71.94}{\kilo\newton\per\meter} \\
	SLU^{\text{sfav}}_{\text{neve}} 		&= \SI{68.92}{\kilo\newton\per\meter} \\
	SLU^{\text{sfav}}_{\text{vento}} 		&= \SI{66.99}{\kilo\newton\per\meter} \\
	SLU^{\text{fav}} 						&= \SI{32.67}{\kilo\newton\per\meter} \\	
	SLE^{\text{rara}}_{\text{cat. B}} 		&= \SI{50.04}{\kilo\newton\per\meter} \\
	SLE^{\text{rara}}_{\text{neve}}			&= \SI{47.94}{\kilo\newton\per\meter} \\
	SLE^{\text{rara}}_{\text{vento}} 		&= \SI{46.93}{\kilo\newton\per\meter} \\
	SLE^{\text{frequente}}_{\text{cat. B}} 	&= \SI{42.85}{\kilo\newton\per\meter} \\
	SLE^{\text{frequente}}_{\text{neve}} 	&= \SI{41.09}{\kilo\newton\per\meter} \\
	SLE^{\text{frequente}}_{\text{vento}} 	&= \SI{40.58}{\kilo\newton\per\meter} \\
	SLE^{\text{quasi perm.}}_{\text{cat. B}}&= \SI{40.55}{\kilo\newton\per\meter}
\end{align*}

\section{Calcolo azioni sulla trave}
\section{Criteri adottati}
%%!TEX root = ../RelazioneStrutturaleMeoliNicola.tex
\begin{figure}[htb]
\centering
\begin{tikzpicture}
\scaling{0.55};
	\point{n1}{00.00}{0};
	\point{n2}{03.00}{0};
	\point{n3}{07.50}{0};
	\point{n4}{11.50}{0};
	\point{n5}{16.50}{0};
	\point{n6}{22.65}{0};
	\point{n7}{26.65}{0};
	\beam{2}{n1}{n2}[0][1];
	\beam{2}{n2}{n3}[1][1];
	\beam{2}{n3}{n4}[1][1];
	\beam{2}{n4}{n5}[1][1];
	\beam{2}{n5}{n6}[1][1];
	\beam{2}{n6}{n7}[1][1];
	\begin{scope}[scale=0.6]
		\support{1}{n1};
		\support{1}{n2};
		\support{1}{n3};
		\support{1}{n4};
		\support{1}{n5};
		\support{1}{n6};
		\support{3}{n7}[90];
	\end{scope};	
	\begin{scope}[color=myGray]
		\lineload{1}{n1}{n2};	
		\notation{1}{n2}{$Q_1=1$}[above=13mm];
	\end{scope}	
	\begin{scope}[color=red]
		\load{2}{n1}[280][130][0.70];
		\load{3}{n2}[130][130][0.70];
		\load{2}{n2}[280][130][0.70];
		\load{3}{n3}[130][130][0.70];
		\load{2}{n3}[280][130][0.70];
		\load{3}{n4}[130][130][0.70];
		\load{2}{n4}[280][130][0.70];
		\load{3}{n5}[130][130][0.70];
		\load{2}{n5}[280][130][0.70];
		\load{3}{n6}[130][130][0.70];
		\load{2}{n6}[280][130][0.70];
		\load{3}{n7}[130][130][0.70];
		\notation{1}{n1}{$x_{11}$}[below=8mm];
		\notation{1}{n2}{$x_{21}$}[below=8mm];
		\notation{1}{n3}{$x_{31}$}[below=8mm];
		\notation{1}{n4}{$x_{41}$}[below=8mm];
		\notation{1}{n5}{$x_{51}$}[below=8mm];
		\notation{1}{n6}{$x_{61}$}[below=8mm];
		\notation{1}{n7}{$x_{71}$}[below=8mm];
	\end{scope}
	\dimensioning{1}{n1}{n2}{-1.5}[$\SI{3.00}{\meter}$];
\end{tikzpicture}
\caption{Metodo delle forze applicato alla prima campata con carico unitario}
\label{fig:Struttura1}
\end{figure}
%!TEX root = ../RelazioneStrutturaleMeoliNicola.tex
\begin{figure}[htb]
\centering
\subfloat[][\emph{DA PENSARCI \label{fig:Struttura2a}}]
{
	\begin{tikzpicture}
	\scaling{0.55};
		\point{n1}{00.00}{0};
		\point{n2}{03.00}{0};
		\point{n3}{07.50}{0};
		\point{n4}{11.50}{0};
		\point{n5}{16.50}{0};
		\point{n6}{22.65}{0};
		\point{n7}{26.65}{0};
		\beam{2}{n1}{n2}[0][1];
		\beam{2}{n2}{n3}[1][1];
		\beam{2}{n3}{n4}[1][1];
		\beam{2}{n4}{n5}[1][1];
		\beam{2}{n5}{n6}[1][1];
		\beam{2}{n6}{n7}[1][1];
		\begin{scope}[scale=0.6]
			\support{1}{n1};
			\support{1}{n2};
			\support{1}{n3};
			\support{1}{n4};
			\support{1}{n5};
			\support{1}{n6};
			\support{3}{n7}[90];
		\end{scope};	
		%
		\begin{scope}[color=red]
			 \lineload{1}{n1}{n2}[1.3][1.3];
			%\lineload{1}{n2}{n3}[1.3][1.3];
			 \lineload{1}{n3}{n4}[1.3][1.3];
			%\lineload{1}{n4}{n5}[1.3][1.3];
			 \lineload{1}{n5}{n6}[1.3][1.3];
			%\lineload{1}{n6}{n7}[1.3][1.3];
		\end{scope};
		\begin{scope}[color=blue]
			%\lineload{1}{n1}{n2}[0.8][0.8];
			 \lineload{1}{n2}{n3}[0.8][0.8];
			%\lineload{1}{n3}{n4}[0.8][0.8];
			 \lineload{1}{n4}{n5}[0.8][0.8];
			%\lineload{1}{n5}{n6}[0.8][0.8];
			 \lineload{1}{n6}{n7}[0.8][0.8];
		\end{scope};
	\end{tikzpicture}
} \\
\subfloat[][\emph{DA PENSARCI \label{fig:Struttura2b}}]
{
	\begin{tikzpicture}
	\scaling{0.55};
		\point{n1}{00.00}{0};
		\point{n2}{03.00}{0};
		\point{n3}{07.50}{0};
		\point{n4}{11.50}{0};
		\point{n5}{16.50}{0};
		\point{n6}{22.65}{0};
		\point{n7}{26.65}{0};
		\beam{2}{n1}{n2}[0][1];
		\beam{2}{n2}{n3}[1][1];
		\beam{2}{n3}{n4}[1][1];
		\beam{2}{n4}{n5}[1][1];
		\beam{2}{n5}{n6}[1][1];
		\beam{2}{n6}{n7}[1][1];
		\begin{scope}[scale=0.6]
			\support{1}{n1};
			\support{1}{n2};
			\support{1}{n3};
			\support{1}{n4};
			\support{1}{n5};
			\support{1}{n6};
			\support{3}{n7}[90];
		\end{scope};	
		%
		\begin{scope}[color=red]
			%\lineload{1}{n1}{n2}[1.3][1.3];
			 \lineload{1}{n2}{n3}[1.3][1.3];
			%\lineload{1}{n3}{n4}[1.3][1.3];
			 \lineload{1}{n4}{n5}[1.3][1.3];
			%\lineload{1}{n5}{n6}[1.3][1.3];
			 \lineload{1}{n6}{n7}[1.3][1.3];
		\end{scope};
		\begin{scope}[color=blue]
			 \lineload{1}{n1}{n2}[0.8][0.8];
			%\lineload{1}{n2}{n3}[0.8][0.8];
			 \lineload{1}{n3}{n4}[0.8][0.8];
			%\lineload{1}{n4}{n5}[0.8][0.8];
			 \lineload{1}{n5}{n6}[0.8][0.8];
			%\lineload{1}{n6}{n7}[0.8][0.8];
		\end{scope};
	\end{tikzpicture}
} \\
\subfloat[][\emph{DA PENSARCI \label{fig:Struttura2c}}]
{
	\begin{tikzpicture}
	\scaling{0.55};
		\point{n1}{00.00}{0};
		\point{n2}{03.00}{0};
		\point{n3}{07.50}{0};
		\point{n4}{11.50}{0};
		\point{n5}{16.50}{0};
		\point{n6}{22.65}{0};
		\point{n7}{26.65}{0};
		\beam{2}{n1}{n2}[0][1];
		\beam{2}{n2}{n3}[1][1];
		\beam{2}{n3}{n4}[1][1];
		\beam{2}{n4}{n5}[1][1];
		\beam{2}{n5}{n6}[1][1];
		\beam{2}{n6}{n7}[1][1];
		\begin{scope}[scale=0.6]
			\support{1}{n1};
			\support{1}{n2};
			\support{1}{n3};
			\support{1}{n4};
			\support{1}{n5};
			\support{1}{n6};
			\support{3}{n7}[90];
		\end{scope};	
		%
		\begin{scope}[color=red]
			 \lineload{1}{n1}{n2}[1.3][1.3];
			 \lineload{1}{n2}{n3}[1.3][1.3];
			%\lineload{1}{n3}{n4}[1.3][1.3];
			 \lineload{1}{n4}{n5}[1.3][1.3];
			%\lineload{1}{n5}{n6}[1.3][1.3];
			 \lineload{1}{n6}{n7}[1.3][1.3];
		\end{scope};
		\begin{scope}[color=blue]
			%\lineload{1}{n1}{n2}[0.8][0.8];
			%\lineload{1}{n2}{n3}[0.8][0.8];
			 \lineload{1}{n3}{n4}[0.8][0.8];
			%\lineload{1}{n4}{n5}[0.8][0.8];
			 \lineload{1}{n5}{n6}[0.8][0.8];
			%\lineload{1}{n6}{n7}[0.8][0.8];
		\end{scope};
	\end{tikzpicture}
} \\
\subfloat[][\emph{DA PENSARCI \label{fig:Struttura2d}}]
{
	\begin{tikzpicture}
	\scaling{0.55};
		\point{n1}{00.00}{0};
		\point{n2}{03.00}{0};
		\point{n3}{07.50}{0};
		\point{n4}{11.50}{0};
		\point{n5}{16.50}{0};
		\point{n6}{22.65}{0};
		\point{n7}{26.65}{0};
		\beam{2}{n1}{n2}[0][1];
		\beam{2}{n2}{n3}[1][1];
		\beam{2}{n3}{n4}[1][1];
		\beam{2}{n4}{n5}[1][1];
		\beam{2}{n5}{n6}[1][1];
		\beam{2}{n6}{n7}[1][1];
		\begin{scope}[scale=0.6]
			\support{1}{n1};
			\support{1}{n2};
			\support{1}{n3};
			\support{1}{n4};
			\support{1}{n5};
			\support{1}{n6};
			\support{3}{n7}[90];
		\end{scope};	
		%
		\begin{scope}[color=red]
			%\lineload{1}{n1}{n2}[1.3][1.3];
			 \lineload{1}{n2}{n3}[1.3][1.3];
			 \lineload{1}{n3}{n4}[1.3][1.3];
			%\lineload{1}{n4}{n5}[1.3][1.3];
			 \lineload{1}{n5}{n6}[1.3][1.3];
			 %\lineload{1}{n6}{n7}[1.3][1.3];
		\end{scope};
		\begin{scope}[color=blue]
			\lineload{1}{n1}{n2}[0.8][0.8];
			%\lineload{1}{n2}{n3}[0.8][0.8];
			%\lineload{1}{n3}{n4}[0.8][0.8];
			\lineload{1}{n4}{n5}[0.8][0.8];
			%\lineload{1}{n5}{n6}[0.8][0.8];
			\lineload{1}{n6}{n7}[0.8][0.8];
		\end{scope};
	\end{tikzpicture}
} \\
\subfloat[][\emph{DA PENSARCI \label{fig:Struttura2e}}]
{
	\begin{tikzpicture}
	\scaling{0.55};
		\point{n1}{00.00}{0};
		\point{n2}{03.00}{0};
		\point{n3}{07.50}{0};
		\point{n4}{11.50}{0};
		\point{n5}{16.50}{0};
		\point{n6}{22.65}{0};
		\point{n7}{26.65}{0};
		\beam{2}{n1}{n2}[0][1];
		\beam{2}{n2}{n3}[1][1];
		\beam{2}{n3}{n4}[1][1];
		\beam{2}{n4}{n5}[1][1];
		\beam{2}{n5}{n6}[1][1];
		\beam{2}{n6}{n7}[1][1];
		\begin{scope}[scale=0.6]
			\support{1}{n1};
			\support{1}{n2};
			\support{1}{n3};
			\support{1}{n4};
			\support{1}{n5};
			\support{1}{n6};
			\support{3}{n7}[90];
		\end{scope};	
		%
		\begin{scope}[color=red]
			 \lineload{1}{n1}{n2}[1.3][1.3];
			%\lineload{1}{n2}{n3}[1.3][1.3];
			 \lineload{1}{n3}{n4}[1.3][1.3];
			 \lineload{1}{n4}{n5}[1.3][1.3];
			%\lineload{1}{n5}{n6}[1.3][1.3];
			 \lineload{1}{n6}{n7}[1.3][1.3];
		\end{scope};
		\begin{scope}[color=blue]
			%\lineload{1}{n1}{n2}[0.8][0.8];
			 \lineload{1}{n2}{n3}[0.8][0.8];
			%\lineload{1}{n3}{n4}[0.8][0.8];
			%\lineload{1}{n4}{n5}[0.8][0.8];
			 \lineload{1}{n5}{n6}[0.8][0.8];
			%\lineload{1}{n6}{n7}[0.8][0.8];
		\end{scope};
	\end{tikzpicture}
} \\
\subfloat[][\emph{DA PENSARCI \label{fig:Struttura2f}}]
{
	\begin{tikzpicture}
	\scaling{0.55};
		\point{n1}{00.00}{0};
		\point{n2}{03.00}{0};
		\point{n3}{07.50}{0};
		\point{n4}{11.50}{0};
		\point{n5}{16.50}{0};
		\point{n6}{22.65}{0};
		\point{n7}{26.65}{0};
		\beam{2}{n1}{n2}[0][1];
		\beam{2}{n2}{n3}[1][1];
		\beam{2}{n3}{n4}[1][1];
		\beam{2}{n4}{n5}[1][1];
		\beam{2}{n5}{n6}[1][1];
		\beam{2}{n6}{n7}[1][1];
		\begin{scope}[scale=0.6]
			\support{1}{n1};
			\support{1}{n2};
			\support{1}{n3};
			\support{1}{n4};
			\support{1}{n5};
			\support{1}{n6};
			\support{3}{n7}[90];
		\end{scope};	
		%
		\begin{scope}[color=red]
			%\lineload{1}{n1}{n2}[1.3][1.3];
			 \lineload{1}{n2}{n3}[1.3][1.3];
			%\lineload{1}{n3}{n4}[1.3][1.3];
			 \lineload{1}{n4}{n5}[1.3][1.3];
			 \lineload{1}{n5}{n6}[1.3][1.3];
			%\lineload{1}{n6}{n7}[1.3][1.3];
		\end{scope};
		\begin{scope}[color=blue]
			 \lineload{1}{n1}{n2}[0.8][0.8];
			%\lineload{1}{n2}{n3}[0.8][0.8];
			 \lineload{1}{n3}{n4}[0.8][0.8];
			%\lineload{1}{n4}{n5}[0.8][0.8];
			%\lineload{1}{n5}{n6}[0.8][0.8];
			 \lineload{1}{n6}{n7}[0.8][0.8];
		\end{scope};
	\end{tikzpicture}
} \\
\subfloat[][\emph{DA PENSARCI \label{fig:Struttura2g}}]
{
	\begin{tikzpicture}
	\scaling{0.55};
		\point{n1}{00.00}{0};
		\point{n2}{03.00}{0};
		\point{n3}{07.50}{0};
		\point{n4}{11.50}{0};
		\point{n5}{16.50}{0};
		\point{n6}{22.65}{0};
		\point{n7}{26.65}{0};
		\beam{2}{n1}{n2}[0][1];
		\beam{2}{n2}{n3}[1][1];
		\beam{2}{n3}{n4}[1][1];
		\beam{2}{n4}{n5}[1][1];
		\beam{2}{n5}{n6}[1][1];
		\beam{2}{n6}{n7}[1][1];
		\begin{scope}[scale=0.6]
			\support{1}{n1};
			\support{1}{n2};
			\support{1}{n3};
			\support{1}{n4};
			\support{1}{n5};
			\support{1}{n6};
			\support{3}{n7}[90];
		\end{scope};	
		%
		\begin{scope}[color=red]
			 \lineload{1}{n1}{n2}[1.3][1.3];
			%\lineload{1}{n2}{n3}[1.3][1.3];
			 \lineload{1}{n3}{n4}[1.3][1.3];
			%\lineload{1}{n4}{n5}[1.3][1.3];
			 \lineload{1}{n5}{n6}[1.3][1.3];
			 \lineload{1}{n6}{n7}[1.3][1.3];
		\end{scope};
		\begin{scope}[color=blue]
			%\lineload{1}{n1}{n2}[0.8][0.8];
			 \lineload{1}{n2}{n3}[0.8][0.8];
			%\lineload{1}{n3}{n4}[0.8][0.8];
			 \lineload{1}{n4}{n5}[0.8][0.8];
			%\lineload{1}{n5}{n6}[0.8][0.8];
			%\lineload{1}{n6}{n7}[0.8][0.8];
		\end{scope};
	\end{tikzpicture}
}
\caption{Disposizione dei carichi sfavorevoli e favorevoli}
\label{fig:Struttura2}
\end{figure}

\section{Momento unitario}

\begin{figure}[p]
\centering
\subfloat[][\emph{Campata 1}]{\includegraphics[width=0.45\textwidth]{../imgExportSage/M1_pUnitario.pdf}} \quad
\subfloat[][\emph{Campata 2}]{\includegraphics[width=0.45\textwidth]{../imgExportSage/M2_pUnitario}} \\
\subfloat[][\emph{Campata 3}]{\includegraphics[width=0.45\textwidth]{../imgExportSage/M3_pUnitario}} \quad
\subfloat[][\emph{Campata 4}]{\includegraphics[width=0.45\textwidth]{../imgExportSage/M4_pUnitario}} \\
\subfloat[][\emph{Campata 5}]{\includegraphics[width=0.45\textwidth]{../imgExportSage/M5_pUnitario.pdf}} \quad
\subfloat[][\emph{Campata 6}]{\includegraphics[width=0.45\textwidth]{../imgExportSage/M6_pUnitario}} \\
\subfloat[][\emph{Somma delle campate}]{\includegraphics[width=0.6\textwidth]{../imgExportSage/Mtot_pUnitario}}
\caption{Diagrammi dei momenti applicando di volta in volta un carico unitario nelle campate e la somma nel diagramma del momento unitario totale}
\label{fig:MomentiUnitari}
\end{figure}
\section{Taglio unitario}
\begin{figure}[htbp]
\centering
\subfloat[][\emph{Campata 1}]{\includegraphics[width=0.45\textwidth]{../imgExportSage/T1_pUnitario.pdf}} \quad
\subfloat[][\emph{Campata 2}]{\includegraphics[width=0.45\textwidth]{../imgExportSage/T2_pUnitario}} \\
\subfloat[][\emph{Campata 3}]{\includegraphics[width=0.45\textwidth]{../imgExportSage/T3_pUnitario}} \quad
\subfloat[][\emph{Campata 4}]{\includegraphics[width=0.45\textwidth]{../imgExportSage/T4_pUnitario}} \\
\subfloat[][\emph{Campata 5}]{\includegraphics[width=0.45\textwidth]{../imgExportSage/T5_pUnitario.pdf}} \quad
\subfloat[][\emph{Campata 6}]{\includegraphics[width=0.45\textwidth]{../imgExportSage/T6_pUnitario}} \\
\subfloat[][\emph{Somma delle campate}]{\includegraphics[width=0.6\textwidth]{../imgExportSage/Ttot_pUnitario}}
\caption{Diagrammi del taglio applicando di volta in volta un carico unitario nelle campate e la somma nel diagramma del taglio unitario totale}
\label{fig:TagliUnitari}
\end{figure}
%
\clearpage	
%%%%%%%%%%%%%%%%%%%%%%%%%%%%%%%%%%%%%%
\begin{landscape}
\begin{figure}[H]
\centering
\subfloat[][\emph{Sovrapposizione dei diagrammi del taglio dovuta alle diverse casistiche dei carico illustrate in figura \ref{fig:Struttura2} \label{fig:TagliA_ULS}}]{\includegraphics[height=0.5\textwidth]{../imgExportSage/ULS_ptot.pdf}} 
\subfloat[][\emph{Inviluppo dei diagrammi del taglio sovrapposti \label{fig:TagliB_ULS}}]{\includegraphics[height=0.5\textwidth]{../imgExportSage/ULS_pInviluppo.pdf}} 
\caption{SLU}
\label{fig:Tagli_ULS}
\end{figure}
\begin{table}[H]
\centering
\caption{boh}
	\begin{tabular}{lS[table-format=3.2]S[table-format=3.2]S[table-format=3.2]S[table-format=3.2]S[table-format=3.2]S[table-format=3.2]S[table-format=3.2]S[table-format=3.2]S[table-format=3.2]S[table-format=3.2]S[table-format=3.2]S[table-format=3.2]S[table-format=3.2]}
		\toprule
		&\multicolumn{1}{c}{N1}&\multicolumn{1}{c}{N1}&\multicolumn{1}{c}{N1}&\multicolumn{1}{c}{N1}&\multicolumn{1}{c}{N1}&\multicolumn{1}{c}{N1}&\multicolumn{1}{c}{N1}&\multicolumn{1}{c}{N1}&\multicolumn{1}{c}{N1}&\multicolumn{1}{c}{N1}&\multicolumn{1}{c}{N1}&\multicolumn{1}{c}{N1}&\multicolumn{1}{c}{N1}\\
		\midrule
		$M^{-}$&999.99&999.99&999.99&999.99&999.99&999.99&999.99&999.99&999.99&999.99&999.99&999.99&999.99\\
		$M^{+}$&999.99&999.99&999.99&999.99&999.99&999.99&999.99&999.99&999.99&999.99&999.99&999.99&999.99\\
		\bottomrule
	\end{tabular}
\end{table}
\end{landscape}
\clearpage
\begin{landscape}
\begin{figure}[H]
\centering
\subfloat[][\emph{Sovrapposizione dei diagrammi del taglio dovuta alle diverse casistiche dei carico illustrate in figura \ref{fig:Struttura2} \label{fig:TagliA_ULS}}]{\includegraphics[height=0.5\textwidth]{../imgExportSage/ULS_Tptot.pdf}} 
\subfloat[][\emph{Inviluppo dei diagrammi del taglio sovrapposti \label{fig:TagliB_ULS}}]{\includegraphics[height=0.5\textwidth]{../imgExportSage/ULS_TpInviluppo.pdf}} 
\caption{SLU}
\label{fig:Tagli_ULS}
\end{figure}
\begin{table}[H]
\centering
\caption{boh}
	\begin{tabular}{lS[table-format=3.2]S[table-format=3.2]S[table-format=3.2]S[table-format=3.2]S[table-format=3.2]S[table-format=3.2]S[table-format=3.2]S[table-format=3.2]S[table-format=3.2]S[table-format=3.2]S[table-format=3.2]S[table-format=3.2]S[table-format=3.2]}
		\toprule
		&\multicolumn{1}{c}{N1}&\multicolumn{1}{c}{N1}&\multicolumn{1}{c}{N1}&\multicolumn{1}{c}{N1}&\multicolumn{1}{c}{N1}&\multicolumn{1}{c}{N1}&\multicolumn{1}{c}{N1}&\multicolumn{1}{c}{N1}&\multicolumn{1}{c}{N1}&\multicolumn{1}{c}{N1}&\multicolumn{1}{c}{N1}&\multicolumn{1}{c}{N1}&\multicolumn{1}{c}{N1}\\
		\midrule
		$M^{-}$&999.99&999.99&999.99&999.99&999.99&999.99&999.99&999.99&999.99&999.99&999.99&999.99&999.99\\
		$M^{+}$&999.99&999.99&999.99&999.99&999.99&999.99&999.99&999.99&999.99&999.99&999.99&999.99&999.99\\
		\bottomrule
	\end{tabular}
\end{table}
\end{landscape}
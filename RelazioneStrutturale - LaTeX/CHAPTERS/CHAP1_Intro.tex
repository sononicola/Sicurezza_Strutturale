%!TEX root = ../RelazioneStrutturaleMeoliNicola.tex
\chapter{Relazione strutturale}
La presente relazione contiene i calcoli della determinazione delle sollecitazioni a momento ed a taglio e il calcolo dello sforzo assiale massimo rispettivamente di una trave e di due pilastri di un edificio. 
Tale edificio è composto da 4 piani di cui 3 fuori terra ed è situato in provincia di Trento ad un'altitudine di \SI{300}{\meter} sopra il livello del mare.

Di seguito si riassumono brevemente le caratteristiche presenti nel progetto strutturale dell'edificio, distribuito all'interno della domanda del committente.
Verranno analizzati i carichi e le diverse stratigrafie di solaio e pareti nei capitoli successivi.
\section{Composizione dell'edificio}
La struttura è a telaio in calcestruzzo armato, caratterizzata da pilastri quadrati di sezione \SI{30}{\centi\meter} e da travi sia a spessore che non, rispettivamente larghe \SI{60}{\centi\meter} ed alte come lo spessore del solaio, e $30 \times \SI{50}{\centi\metre}$.
Nel caso di queste ultime il momento di inerzia $J$ vale $\frac{b\cdot h^3}{12}=\frac{30\cdot50^3}{12}=\SI{3.125e-3}{\meter^4}$ ed assumendo un modulo elastico $E$ di \SI{31476}{\mega\pascal} si ottiene un $EJ=\SI{98362.5}{\kilo\newton\per\meter\squared}$

I solai strutturali sono di tipo a lastre tralicciate Predalle tra il piano interrato e il piano terra e nel solaio di copertura. 
Il peso ultimato di tale solaio di \SI{3.6}{\kilo\newton\per\square\meter}.
I solai tra piano terra e piano primo e tra pianbo primo e piano secondo sono di tipo a travetti e latero-cemento.
Il peso ultimato di tale solaio di \SI{3.20}{\kilo\newton\per\square\meter}.

I solai dei piano intermedi sono costituiti da un pacchetto con calcestruzzo alleggerito, massetto di allettamento, intonaco e pavimento in ceramica.
Il solaio nella zona del terrazzo è costituito da isolante, impermealizzazione, massetto in calcestruzzo e pavimento.
Il solaio di copertura è costituito da isolante, massetto in calcestruzzo alleggerito, impermealizzazione e ghiaia.

Le pareti divisorie interne sono costituite da muratura in laterizio e intonaco in ambo i lati.
Le pareti perimetrali sono costituite da muratura in laterizio, cappotto esterno e intonaco.

Il piano terra è adibito a negozi, il piano primo ad uffici aperti al pubblico, il piano secondo a civile abitazione.
Il piano interrato è adibito a garage.

\section{Normativa di riferimento e coefficienti utilizzati}
Sono state utilizzate le normative attualmente in vigore in Italia, in particolare 
\begin{itemize}
\item \norma{Decreto Ministeriale 17 gennaio 2018}, d'ora in poi chiamato \norma{NTC2018}
\item \norma{Circolare 21 gennaio 2019 n. 7, C.S.LL.PP}, d'ora in poi chiamata \norma{circolare}
\end{itemize}
Nel calcolo delle combinazioni di carico agli SLU e SLE sono stati utilizzati i coefficienti di combinazione e i coefficienti parziali in \normaref{Tab\,2.5.I} ed in \normaref{Tab\,2.6.I} riportati dalle \norma{NTC2018}.
Vengono riportati per comodità in questo capitolo.
\clearpage
\begin{figure*}[htb]
\subfloat{\includegraphics[width=0.55\textwidth,valign=t]{IMG/tab2-5-I.pdf}} 
\subfloat{\includegraphics[width=0.45\textwidth,valign=t]{IMG/tab2-6-I.pdf}}\\
\subfloat{\includegraphics[width=0.55\textwidth]{IMG/tab2-5-I-2.pdf}} 
\end{figure*}
\section{Criteri stilistici adottati nella presente relazione}
Per maggior chiarezza di lettura si vogliono qui esplicare i diversi tipi di carattere che sono presenti all'interno di queste pagine.
\paragraph*{Numeri di figure e tabelle nelle didascalie} Per distinguere meglio i nomi che figure e tabelle hanno: con il \textbf{Testo in nero} sono riportati direttamente le immagini prese dalle rispettive norme; con il \textbf{\textcolor{pantone186}{Testo in rosso}} sono riportati figure e tabelle create appositamente per questa relazione.
\paragraph*{Richiami ad elementi presenti nelle normative} Con \normaref{questo carattere} utilizzato all'interno dei testi che seguiranno si intende richiamare figure, tabelle o paragrafi presenti nelle normative (richiamate con \norma{questo carattere}).
Con il carattere normale vengono invece richiamati gli elementi creati appositamente in questa relazione (figure, tabelle o formule).
\paragraph*{Separatore numeri decimali} Nel caso di numero avente un'unità di misura si è utilizzata la virgola. 
Se il numero è adimensionale, come per esempio i coefficienti, si è utilizzato il punto.

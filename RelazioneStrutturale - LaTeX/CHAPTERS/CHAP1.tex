%!TEX root = ../RelazioneStrutturaleMeoliNicola.tex
\chapter{Relazione strutturale}
\chapter{Analisi dei carichi}
\section{Trave P - Vano scale -- Piano primo}
\subsection{Terrazzo}
\paragraph*{Carichi permanenti G1}
\paragraph*{Carichi permanenti non strutturali G2}
\paragraph*{Categoria B - balconi} La normativa prevede un carico distribuito pari a 
\paragraph*{Neve}
L'edificio è ubicato in zona 1 ad una quota superiore a \SI{200}{\meter} pertanto il valore di riferimento del carico della neve al suola risulta pari a
\[
	q_{sk}=1.39 \, [1+(a_s/728)^2] = \SI{1.626}{\kilo\newton\per\square\meter}
\]

Si assume che l'edificio sia in zona normale di vento. Pertanto $C_E$ risulta pari a $1$.
Si assume un coefficiente termico $C_t = 1$ in quanto è assente uno specifico studio riguardo la perdita di calore della costruzione. 

Per il calcolo del coefficiente di forma $\mu_i$ la normativa prevede due possibili casi dovuti alla vicinanza della copertura a costruzioni più alte in quanto si genera un accumolo di neve.
Il primo caso prevede $\mu_1=0.8$ ed è costante data la copertura piana. Nel secondo $\mu_2$ è la somma tra il contributo $\mu_s$ dello scivolamento della neve dalla copertura al piano superiore e pertanto è nullo essendo piana anch'essa. 
E il contributo $\mu_w$ dovuto al vento che redistribuisce la neve. 
Questo vale 
\[\mu_w=\frac{b_1 + b_2}{2\,h}=\frac{18+6}{2\cdot6.2}=1.935\] e nel quale si è considerata come $b_2$ fissa a favore di sicurezza.

Essendo $l_s=2\,h>b_2$ il coefficiente $\mu$ deve essere valutato come interpolazione tra i due casi, risultando quindi pari a $(0.8 + 1.935)/2 = 1.368$

Il carico dovuto alla neve sul terrazzo risulta infine pari a 
\[q_s = q_{sk} \cdot C_E \cdot C_t \cdot \mu = \SI{1.626}{} \cdot 1 \cdot 1 \cdot 1.368 = \SI{2.224}{\kilo\newton\per\square\meter}\]

\paragraph*{Vento}
\subsection{Interno}
\subsection{Pareti perimetrali}

\section{Combinazioni di carico}
Al fine di trovare le azioni più incisive nel caso di carico massimo e di carico minimo, si sono valutate le azioni sfavorevoli e favorevoli con diverse disposizione nelle campate. 
Si elencheranno qui le diverse possibili combinazioni di carico agli stati limite ultimi e di esercizio.
\begin{align}
	\begin{split}
	SLU^{\text{sfav}}_{\text{sovraccarico}} &= \gamma_{G1}\cdot G1 + \gamma_{G2} \cdot G2 + \gamma_{sovr} \cdot (Q_{terr.} + Q_{int.}) + \gamma_{neve}\cdot Q_{neve}\cdot\psi_{02} + \gamma_{vento}\cdot Q_{vento} \cdot \psi_{03}  \\
											&= altro \\
											&= risultato
	\end{split} \\
	\begin{split}
	SLU^{\text{sfav}}_{\text{neve}} &= qualcosa \\
									&= altro \\
									&= risultato
	\end{split} \\
	\begin{split}
	SLU^{\text{sfav}}_{\text{vento}}&= qualcosa \\
									&= altro \\
									&= risultato
	\end{split} \\
	SLU^{\text{fav}}	
\end{align}


\section{Pilastro P13}

\chapter{Calcolo azioni sulla trave}
\section{Criteri addottati}
\section{Momento}

\appendix
\chapter{Codice risoluzione trave}

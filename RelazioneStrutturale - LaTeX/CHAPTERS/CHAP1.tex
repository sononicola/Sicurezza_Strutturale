%!TEX root = ../RelazioneStrutturaleMeoliNicola.tex
\chapter{Relazione strutturale}
\chapter{Trave P13 - P18 - Vano scale -- Piano primo}
\section{Analisi dei carichi trave}
\subsection{Peso proprio trave}
La trave è di sezione triangolare $30 \times \SI{50}{\centi\metre}$ in calcestruzzo armato. 
La normativa suggerisce di utilizzare un peso specifico $\gamma_{CLS}$ pari a \SI{25.0}{\kilo\newton\per\meter\cubed}. 
Pertanto il carico lineare risulta 
\[
	G_1^{trave} = 0.3 \cdot 0.5 \cdot 25 = \SI{3.75}{\kilo\newton\per\square}
\]

\subsection{Terrazzo}
\paragraph*{Carichi permanenti G1}
Come da progetto il peso del solaio ultimato in travetti tralicciati in latero cemento è pari a $g_1^{ter.}\SI{3.20}{\kilo\newton\per\meter}$.
\paragraph*{Carichi permanenti non strutturali G2}
Il carico distribuito su superficie permanente non strutturale agente sul terrazzo è la somma dei singoli carichi degli strati che compongono la stratigrafia presente.
\begin{center}
\begin{tabular}{lS[table-format=2.1]S[table-format=1.2]S[table-format=1.3]}
	\toprule
	\multirow{2}{*}{Strato} & \multicolumn{1}{c}{Peso specifico} & \multicolumn{1}{c}{Spessore}& \multicolumn{1}{c}{$g_{2,k}$}\\
    	   & \multicolumn{1}{c}{$\left[\SI{}{\kilo\newton\per\meter\cubed}\right]$} & \multicolumn{1}{c}{$\left[\SI{}{\meter}\right]$}& \multicolumn{1}{c}{$\left[\SI{}{\kilo\newton\per\square\meter}\right]$}\\
	\midrule
	Isolante 	             & 0.5  & 0.15 & 0.075 \\
	Massetto calcestruzzo 	 & 24.0 & 0.06 & 1.44  \\
	Pavimento 	             &      &      & 0.50  \\
	Intonaco intradosso 	 & 20.0 & 0.01 & 0.20 \\
	\midrule
	Totale $g_2^{ter.} =$ &&&2.215\\
	\bottomrule
\end{tabular}
\end{center}
\paragraph*{Categoria B - balconi} La normativa prevede un carico distribuito pari a 
\paragraph*{Neve}
L'edificio è ubicato in zona 1 ad una quota superiore a \SI{200}{\meter} pertanto il valore di riferimento del carico della neve al suola risulta pari a
\[
	q_{sk}=1.39 \, [1+(a_s/728)^2] = \SI{1.626}{\kilo\newton\per\square\meter}
\]

Si assume che l'edificio sia in zona normale di vento. Pertanto $C_E$ risulta pari a $1$.
Si assume un coefficiente termico $C_t = 1$ in quanto è assente uno specifico studio riguardo la perdita di calore della costruzione. 

Per il calcolo del coefficiente di forma $\mu_i$ la normativa prevede due possibili casi dovuti alla vicinanza della copertura a costruzioni più alte in quanto si genera un accumolo di neve.
Il primo caso prevede $\mu_1=0.8$ ed è costante data la copertura piana. Nel secondo $\mu_2$ è la somma tra il contributo $\mu_s$ dello scivolamento della neve dalla copertura al piano superiore e pertanto è nullo essendo piana anch'essa. 
E il contributo $\mu_w$ dovuto al vento che redistribuisce la neve. 
Questo vale 
\[\mu_w=\frac{b_1 + b_2}{2\,h}=\frac{18+6}{2\cdot6.2}=1.935\] e nel quale si è considerata come $b_2$ fissa a favore di sicurezza.

Essendo $l_s=2\,h>b_2$ il coefficiente $\mu$ deve essere valutato come interpolazione tra i due casi, risultando quindi pari a $(0.8 + 1.935)/2 = 1.368$

Il carico dovuto alla neve sul terrazzo risulta infine pari a 
\[
q_s = q_{sk} \cdot C_E \cdot C_t \cdot \mu = \SI{1.626}{} \cdot 1 \cdot 1 \cdot 1.368 = \SI{2.224}{\kilo\newton\per\square\meter}
\]
\paragraph*{Vento}
\subsection{Interno}
\paragraph*{Carichi permanenti strutturali G1}
\e presente il medesimo solaio strutturale del terrazzo, pertanto $g_1^{sol.}\SI{3.20}{\kilo\newton\per\meter}$.
\paragraph*{Carichi permanenti non strutturali G2} Sono costituiti dal pacchetto non strutturale della stratigrafia del solaio e dalle pareti divisorie interne. 
Per quanto riguarda le pareti divisorie interne il \normaref{capitolo 3.1.3} delle \norma{ntc} permette di spalmare il peso delle pareti interne in un carico distribuito su tutta la superficie.
\begin{center}
\begin{tabular}{lS[table-format=2.1]S[table-format=1.2]S[table-format=1.3]}
	\toprule
	\multirow{2}{*}{Strato} & \multicolumn{1}{c}{Peso specifico} & \multicolumn{1}{c}{Spessore}& \multicolumn{1}{c}{$g_{2,k}$}\\
    	   & \multicolumn{1}{c}{$\left[\SI{}{\kilo\newton\per\meter\cubed}\right]$} & \multicolumn{1}{c}{$\left[\SI{}{\meter}\right]$}& \multicolumn{1}{c}{$\left[\SI{}{\kilo\newton\per\square\meter}\right]$}\\
	\midrule
	Tramezze in laterizio 	 	 & 8.00 & 0.08 & 0.64 \\
	Intonaco interno 	     	 & 20.0 & 0.01 & 0.2 \\
	Intonaco esterno	         & 20.0 & 0.01 & 0.2 \\
	\midrule
	Totale $=$   				 &      &      & 1.04 \\
	\bottomrule
\end{tabular}
\end{center}
L'altezza delle pareti corrisponde all'altezza di interpiano meno lo spessore del solaio, il che risulta $\SI{3.10}{} - \SI{0.25}{} = \SI{2.85}{\meter}$.
Il carico lineare delle pareti interne diviene quindi \SI{2.964}{\kilo\newton\per\meter}.
Utilizzando la normativa si ottiene così un carico di \SI{1.20}{\kilo\newton\per\square\meter}.

Unendo tutti i contributi si ha 
\begin{center}
\begin{tabular}{lS[table-format=2.1]S[table-format=1.2]S[table-format=1.3]}
	\toprule
	\multirow{2}{*}{Strato} & \multicolumn{1}{c}{Peso specifico} & \multicolumn{1}{c}{Spessore}& \multicolumn{1}{c}{$g_{2,k}$}\\
    	   & \multicolumn{1}{c}{$\left[\SI{}{\kilo\newton\per\meter\cubed}\right]$} & \multicolumn{1}{c}{$\left[\SI{}{\meter}\right]$}& \multicolumn{1}{c}{$\left[\SI{}{\kilo\newton\per\square\meter}\right]$}\\
	\midrule
	Sottofondo CLS alleggerito 	 & 16.0 & 0.08 & 1.28 \\
	Massetto allettamento 	     & 24.0 & 0.06 & 1.44 \\
	Pavimento ceramica 	         &      &      & 0.50 \\
	Intonaco intradosso 	     & 20.0 & 0.01 & 0.20 \\
	Pareti interne distribuite   &      &      & 1.20 \\
	\midrule
	Totale $g_2^{sol.} =$        &      &      & 4.62 \\
	\bottomrule
\end{tabular}
\end{center}
\subsection{Pareti perimetrali}
\begin{center}
\begin{tabular}{lS[table-format=2.1]S[table-format=1.2]S[table-format=1.3]}
	\toprule
	\multirow{2}{*}{Strato} & \multicolumn{1}{c}{Peso specifico} & \multicolumn{1}{c}{Spessore}& \multicolumn{1}{c}{$g_{2,k}$}\\
    	   & \multicolumn{1}{c}{$\left[\SI{}{\kilo\newton\per\meter\cubed}\right]$} & \multicolumn{1}{c}{$\left[\SI{}{\meter}\right]$}& \multicolumn{1}{c}{$\left[\SI{}{\kilo\newton\per\square\meter}\right]$}\\
	\midrule
	Muratura in laterizio 	 	 & 10   & 0.30 & 3 \\
	Intonaco interno 	     	 & 20.0 & 0.01 & 0.2 \\
	Cappotto esterno	         & 0.20 & 0.12 & 0.024 \\
	\midrule
	Totale $=$   				 &      &      & 3.224 \\
	\bottomrule
\end{tabular}
\end{center}
Sono agenti direttamente con un carico lineare al di sopra della trave e si estendono per una altezza pari a quella di interpiano meno la trave.
Ovvero $\SI{3.10}{} - \SI{0.5}{} = \SI{2.60}{\meter}$.
\[
G_2^{pareti}=\SI{3.224}{\kilo\newton\per\square\meter} \cdot \SI{2.60}{\meter} = \SI{8.382}{\kilo\newton\per\meter}
\]
\subsection{Totale carichi agenti sulla trave}
Vengono ora moltiplicati i risultati appena trovati per le relative lunghezze di influenza. Si divide il problema in due zone: A e B come mostrato in FIGURA DA METTERE. 
Si è considerato il carico uniformemente distribuito nelle due travi adiacenti, pertanto le lunghezze saranno pari alla metà della distanza tra gli interassi del solai. 
Nella zona A è quindi pari a \SI{3.00}{\meter} nel terrazzo e \SI{2.50}{\meter} nel solaio interno. 
Nella zona B è rispettivamente pari a \SI{1.75}{\meter} e \SI{2.50}{\meter}.

I carichi a metro lineare sotto riportati tengono conto di tali lunghezze e sono stati combinati con le relativi azioni agenti sulla trave.
\paragraph*{Zona A} 
\[
\begin{split}
G_1^A &=  g_1^{ter.}\cdot L^{ter.} + g_1^{sol.}\cdot L^{sol.} + G_1^{trave} \\
&= \SI{3.20}{\kilo\newton\per\square\meter}\cdot\SI{3.00}{\meter} + \SI{3.20}{\kilo\newton\per\square\meter}\cdot\SI{2.50}{\meter} + \SI{3.75}{\kilo\newton\per\meter} \\
&= \SI{21.35}{\kilo\newton\per\meter} \\
G_2^A &= g_2^{ter.}\cdot L^{ter.} + g_2^{sol.}\cdot L^{sol.} + G_2^{pareti} \\
&= \SI{2.215}{\kilo\newton\per\square\meter}\cdot\SI{3.00}{\meter} + \SI{4.62}{\kilo\newton\per\square\meter}\cdot\SI{2.50}{\meter} + \SI{8.382}{\kilo\newton\per\meter} \\
&= \SI{26.58}{\kilo\newton\per\meter}
\end{split}
\]
\paragraph*{Zona B} I carichi su superficie sono gli stessi della zona A e cambiano solo le lunghezze di riferimento. 
Si ha infine
\begin{align*}
G_1^B &= \SI{17.35}{\kilo\newton\per\meter}\\
G_2^B &= \SI{23.81}{\kilo\newton\per\meter}
\end{align*}
\section{Combinazioni di carico}
Al fine di trovare le azioni più incisive nel caso di carico massimo e di carico minimo, si sono valutate le azioni sfavorevoli e favorevoli con diverse disposizione nelle campate. 
Si elencheranno qui le diverse possibili combinazioni di carico agli stati limite ultimi e di esercizio.
\begin{align}
	\begin{split}
	SLU^{\text{sfav}}_{\text{cat. B}} &= \gamma_{G1}\cdot G_1 + \gamma_{G2} \cdot G_2 + \gamma_{sovr} \cdot (Q_{terr.} + Q_{int.}) + \gamma_{neve}\cdot Q_{neve}\cdot\psi_{02} + \gamma_{vento}\cdot Q_{vento} \cdot \psi_{03}  \\
											&= altro \\
											&= risultato
	\end{split} \\
	\begin{split}
	SLU^{\text{sfav}}_{\text{neve}} &= qualcosa \\
									&= altro \\
									&= risultato
	\end{split} \\
	\begin{split}
	SLU^{\text{sfav}}_{\text{vento}}&= qualcosa \\
									&= altro \\
									&= risultato
	\end{split} \\
	SLU^{\text{fav}}	
\end{align}

\section{Calcolo azioni sulla trave}
\section{Criteri addottati}
\section{Momento}
\chapter{Pilastro P27}
\chapter{Pilastro P36}



\appendix
\chapter{Codice risoluzione trave}